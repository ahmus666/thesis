% Wichtige Pakete und Grundeinstellungen, verwendet Dokumentenklasse "scrbook" 
\documentclass[
		chapterprefix=false, 
		12pt, 
		a4paper, 
		twoside, 
		parskip=half, 
		listof=totoc, 
		bibliography=totoc, 
		numbers=noendperiod, 
		captions=tableheading
]{scrbook}

%Anpassung der Seitenränder 
%% Geometrisches ...
\usepackage{geometry}
\makeatletter
\setlength\paperheight     {297mm}
\setlength\paperwidth  	   {210mm}
\setlength\headheight      {2ex}
\setlength\headsep         {4ex} %2
\setlength\footskip        {35pt} %25
\setlength\textwidth       {150mm}
\setlength\textheight      {240mm} %245
\setlength{\@tempdima}     {\paperwidth}
\addtolength{\@tempdima}   {-2in}
\addtolength{\@tempdima}   {-\textwidth}
\setlength\oddsidemargin   {0.5\@tempdima}
\setlength\evensidemargin  {\oddsidemargin}
\setlength{\@tempdima}     {\paperheight}
\addtolength{\@tempdima}   {-3in}
\addtolength{\@tempdima}   {-\textheight}
\setlength\topmargin       {.5\@tempdima}
\setlength\footnotesep     {12\p@}
\setlength{\skip\footins}  {10\p@ \@plus 2\p@ \@minus 4\p@}
\setlength{\marginparsep}  {1pt}
\setlength{\marginparwidth}{20mm}
\makeatother 


%% Unterstützung von Umlauten und anderen Sonderzeichen (UTF-8)
\usepackage{lmodern}
\usepackage[utf8]{inputenc}
\usepackage[T1]{fontenc}

%% Allgemeine Anpassungen 
\usepackage{scrhack} %Tweaks für scrbook
\usepackage[automark,headsepline]{scrlayer-scrpage} %Anpassung von Kopf- und Fußzeile
\usepackage{paralist}  %Kompakte Listen
\usepackage[onehalfspacing]{setspace} %Zeilenabstand 1,5
\usepackage[stretch=10]{microtype} %Verbesserte Darstellung der Buchstaben zueinander
\usepackage{float} %Unterstützung fester Positionierung  (H)

%Glossar, Stichworverzeichnis (Akronyme als eigene Liste)
%\usepackage[toc, acronym]{glossaries} 

%% Mathematik:
\RequirePackage{amssymb, amsmath}
\RequirePackage{ifthen}
\RequirePackage{ae,pifont}

%% Erweiterte Tabellen:
\usepackage{array, tabularx}
\usepackage{multirow}
\usepackage{longtable} %Umbrüche in Tabellen

%% Grafiken:
% Definition eigener Farben 
\usepackage[table,xcdraw]{xcolor}
\usepackage{tikz} % Vektorgrafiken
\usepackage{graphicx} % Pixelgrafiken (wie jpg, png, etc.)


%Verknüpfungen im Dokument, Links werden durch "hidelink" nicht explizit hervorgehoben
\usepackage[hidelinks,german]{hyperref}

%% Erstellung von Literaturverzeichnissen mit Biber
\usepackage{csquotes} %wird von biber benötigt
\usepackage[style=alphabetic, backend=biber, bibencoding=ascii, natbib=true]{biblatex}
\addbibresource{references/references.bib}

